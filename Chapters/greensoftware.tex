\section{Green Software}
\label{sc:greensoftware}


Since the exploding of the \gls{it} in all areas of our activity, offering great benefits, convenience, opportunities and irreversibly transforming businesses and society, it has also been contributing to environmental problems \cite{makingitgreen}.

With an increase of concerns with energy consumption in all areas, it grows the use of Green computing, also known as Green \gls{it} in the computer science area. Green computing is the area that study and practice environmentally friendly and sustainable computing. The goals of this area are to reduce and understand the energy consumption of different technologies, Software, or hardware and which choice can we make to reduce energy consumption. Green Software is a sub-area of green computing that mains goals are to reduce energy consumption through software analysis and optimization. Therefore, developing green software can contribute significantly to preserve the environment and reduce energy consumption globally. 


%Parte dos related work
\subsection{Related Work}


Even though green software isn't as popular as it should be a wide range of studies have already been made towards introducing and creating more energy-friendly and energy-aware approaches. At the moment, the work done on energy consumption is clear proof of the paradigm shift in the creation of software. There is a lot of examples of this shift on energy awareness on software development, examples of this is the work made by \citeauthor{javaenergy}~(\citeyear{javaenergy}), that analyze the energy consumption of the different \gls{jcf}, present an energy optimization approach for Java programs: based on calls to \gls{jcf} methods in the source code of a program and define a green ranking for Java Collections. In \citeauthor{10.1145/3125374.3125382}~(\citeyear{10.1145/3125374.3125382}), authors define a ranking of energy efficiency in the programming language in ten well-known programming languages by running a set of computing problems in each language and monetize the energy consumption.In the area of programming languages, there are other examples of studies \cite{7965265,PEREIRA2020110463,LIMA2019554,pereira2017energy,carccao2014measuring,7965316}. Other studies can be found in different areas like the mobile area, a study developed by \citeauthor{10.1007/978-3-319-11863-5_6}~(\citeyear{10.1007/978-3-319-11863-5_6}) aimed at detecting Anomalous Energy Consumption in Android Applications. \citeauthor{8816732}~(\citeyear{8816732}) developed the GreenSource infrastructure: a large body of open-source Android applications tailored for energy analysis and optimization. There are a lot more examples in the mobile area \cite{10.1145/3387905.3388600,9054858,10.1109/MSR.2019.00034,Rua2019TowardsUM,cruz2019catalog,cruzenery,7972807}.

\gls{dbms} doesn't escape the concerns and care of the green movement, \citeauthor{agrawal2008claremont}~(\citeyear{agrawal2008claremont}) made an early approach concerning energy consumption on database systems and his report, he said people should take into consideration designing power-aware \gls{dbms} that limit energy costs without sacrificing scalability.
\citeauthor{HarizopoulosEnergy}~(\citeyear{HarizopoulosEnergy}) focuses on finding software-level optimization characteristics that might improve the energy efficiency of Data Management Systems. 
In work provide by \citeauthor{wang2011survey}~(\citeyear{wang2011survey}), he presents a survey about energy efficiency in data management operations.
%There is another study's related to energy consumption on \gls{dbms} \cite{citar}.

Although most of the study's show previously focus more on the hardware base premises. There is other more focus on software like,
\citeauthor{5447840}~(\citeyear{5447840}) presented a solution of power-performance tradeoffs on \gls{dbms} where are results show that exists attractive tradeoffs between average-power and time-efficiency. \citeauthor{xupet}~(\citeyear{xupet}) also proposes query optimization intending to reduce energy consumption. Additionally, To reduce the peak of energy usage in database management systems \citeauthor{KunjirPeakPower}~(\citeyear{KunjirPeakPower}) proposed several alternatives. In \citeauthor{RODRIGUEZMARTINEZ2011112}~(\citeyear{RODRIGUEZMARTINEZ2011112})'s paper, he presents an empirical methodology to estimate the power and energy cost of database operations, on a similar context \citeauthor{6738985}~(\citeyear{6738985})  developed a work related to the prediction of the energy consumption of join queries. Later, \citeauthor{gonçalvesbelo}~(\citeyear{gonçalvesbelo}) redesigned the \gls{dbms} execution plan to include both the average energy consumption value for the most common database operators and the total query energy estimation. Afterward, these authors made similar work on a different domain about measure energy consumption for green star-queries in data warehousing systems \cite{7396507}. After this, as a simple guideline for reducing the energy consumption of a given query within a relational \gls{dbms},  \cite{guimaraes2016some} devised a collection of heuristics. There are studies for \gls{nosql}, one example of that are \citeauthor{duarte2017evaluating}~(\citeyear{duarte2017evaluating}) work that focuses on energy consumption on document stores based systems. \citeauthor{Authenticus:P-00P-QKR}~(\citeyear{Authenticus:P-00P-QKR}) developed a work about query energy consumption comparing the energy efficiency between a relational and a non-relational \gls{dbms}. Another study related to the previous one is \citeauthor{mahajan2016energy}~(\citeyear{mahajan2016energy}) work that compares energy consumption between different \gls{dbms} like MySQL, MongoDB, and Cassandra on a query level.



In this work, we extend our experience in energy consumption evaluation on different databases. Where, unlike the other studies show we want to evaluate and compare the energy efficiency between different \gls{dbms} in a realistic environment.







