\chapter{Conclusion and Future Work}
\label{cha:conclusion}

	
	This thesis fits in the green software area, which is a crucial topic in today's world which over the years, energy-efficient of large-scale computing has become more relevant with the high growth in the volume of data processed through cloud services. Since \gls{dbms} can work as a cloud service, it is essential to understand and study their consumption.
	
	This dissertation ends with this chapter, which focuses on the main conclusions taken from the results benchmark of \gls{dbms}.	All things said we divided this chapter into two sections. First, Section 1 starts with a summary of the study carried out during this thesis and a brief answer to the research questions formulated in Chapter 3. Finally, Section 2 makes some ideas about research for the future on this topic and some limitations in this work.
	

\section{Final Considerations}
\label{sc:finalcons}
%Resumir dos objetivos
In summary, the main aim of this master's thesis was to compare various \gls{dbms} applications in the most practical environment scenario possible.

%Motivação por traz disto
The impulse that drove this project was the concern about energy awareness in developing software and the lack of knowledge in \gls{dbms} energy consumption. One of the motives to choosing \gls{dbms} and not other software was the current studies and practices on databases emphasizing performance more than energy efficiency. Also, this area presents solutions for improving energy consumption in a data center when dealing with \gls{dbms}.
%O Que se fez

%Responder as RQ

Table \ref{tb:conclusion} is a summary view of the results obtained in this thesis. There we present the results into a classification table of each SDGB in each scenario.
\begin{table}[H]
\centering
\begin{tabular}{|c|c|c|c|c|}
\hline
Scenarios          & \multicolumn{1}{l|}{MySQL} & \multicolumn{1}{l|}{Postgres} & \multicolumn{1}{l|}{MariaDB} & Redis \\ \hline
\multicolumn{5}{|c|}{Static Virtual User (1 VU)}                                                                       \\ \hline
Energy Consumption & 4                          & 2                             & 3                            & 1     \\ \hline
Performance TPM    & 4                          & 2                             & 3                            & 1     \\ \hline
Performance NOPM   & 4                          & 1                             & 3                            & 2     \\ \hline
Energy per TPM     & 4                          & 2                             & 3                            & 1     \\ \hline
Energy per NOPM    & 4                          & 1                             & 3                            & 2     \\ \hline
\multicolumn{5}{|c|}{Multi Virtual Users}                                                                             \\ \hline
Energy Consumption & 2                          & 3                             & 4                             & 1     \\ \hline
Performance TPM    & 4                          & 3                             & 2                            & 1     \\ \hline
Performance NOPM   & 4                          & 3                             & 2                            & 1     \\ \hline
Energy per TPM     & 4                          & 3                             & 2                            & 1     \\ \hline
Energy per NOPM    & 4                          & 3                             & 2                            & 1     \\ \hline
\end{tabular}
\caption{Classification of each DBMS in each Scenario}
\label{tb:conclusion}
\end{table}

After analyzing these results, it is possible to answer the three research questions presented in section \ref{sc:rq}: 



\begin{itemize}
  \item \textbf{RQ1}:\textit{Which \acrlong{dbms} is the most energy-efficient?}
	Here the conclusion is that with only one virtual user, Redis is the most energy-efficient on a general note by far. If we only compare Relational \gls{dbms}, the best is Postgres, follow by MariaDB then MySQL. On a more specific level like Package/\gls{cpu}, due to its non-relational nature, Redis is less energy-efficient, while it is, due to the same reason, the most efficient on the Disk level.
    

 \item \textbf{RQ2}:\textit{Which \acrlong{dbms} has the best energy versus runtime tradeoff?}
Here is a bit different since this question is about which database has the best performance per energy consumption. Here it has two observations, the first one of \gls{tpm} and the more reliable \gls{nopm}.
On energy consumption per \gls{tpm}, Redis is the most efficient in all subcomponents followed by Postgres, MariaDB, and the worst MySQL.
Here at all levels, Postgres is the one with the lowest energy consumption, followed by Redis, MariaDB, and MySQL.

 
%\discuss{Não será esta a pergunta? \textit{Which Database Management Systems has the best energy versus runtime tradeoff?} Parece-me Bem}
 
  \item \textbf{RQ3}:\textit{How does the increasing number of users of the \acrlong{dbms} impact their energy consumption? } In this final Research Question, there are two topics to be covered:
The first is what \gls{dbms} has better scalability in terms of energy consumption, where MariaDB is by far the worst energy-efficient \gls{dbms} with an increase of users, Postgres maintains similar energy consumption, MySQL and Redis improve their energy efficiency.
The other topic is energy consumption per performance scalability in both energy consumption per \gls{tpm} and energy consumption per \gls{nopm}, which has similar results, which Redis is who improves most, followed by MariaDB, Postgres, and MySQL.
\end{itemize}


\section{Future Work}

The presented study in this master thesis, as initially proposed, reduces some of the lack of knowledge about energy efficiency in \gls{dbms}.  Even though this study furthered the advancement of Energyware Engineering in \gls{dbms}, there is still work left to do.  

In terms of work that could boost this study would be doing the same benchmark with longer times. Even though we did tests for 5, 10, and 30 minutes, the objective would be doing it for an even longer time, for example, 24 hours. Additionally, extend this study to all \gls{dbms} available on HammerDB but different types of \gls{dbms} not available in HammerDB.

Another path to this study would be to do this study with different hardware and compare the results. Examples of this would be to use a different size of \gls{ram}, \acrshort{ssd}, or a \gls{cpu}.

Since the benchmark applied here is a standard benchmark, it would be interesting to explore the \gls{dbms} energy consumption administered to different benchmark software and workloads. With this, it would bring a vision of \gls{dbms} energy consumption applied to distinct environments. The last suggestion would be doing a benchmark but applying energy-aware query optimization in this benchmark as a custom script and with this understand the effect of this optimization on an energy-efficiency and performance.

\begin{comment}\paragraph{RQ1:} \textit{Which \acrlong{dbms} is the most energy-efficient?}

	Here the conclusion is that with only one virtual user, Redis is the most energy-efficient on a general note by far. If we only compare Relational \gls{dbms}, the best is Postgres, follow by MariaDB then MySQL. On a more specific level like Package/\gls{cpu}, due to its non-relational nature, Redis is less energy-efficient, while it is, due to the same reason, the most efficient on the Disk level.
    
     \paragraph{RQ2:}\textit{Which \acrlong{dbms} has the best energy versus runtime tradeoff?}

Here is a bit different since this question is about which database has the best performance per energy consumption. Here it has two observations, the first one of \gls{tpm} and the more reliable \gls{nopm}.

On energy consumption per \gls{tpm}, Redis is the most efficient in all subcomponents followed by Postgres, MariaDB, and the worst MySQL.

Here at all levels, Postgres is the one with the lowest energy consumption, followed by Redis, MariaDB, and MySQL.

     \paragraph{RQ3:} Which Database Management Systems are the most energy-efficient with a different number of users?

In this final RQ3, there are two topics to be covered. 
The first is what \gls{dbms} has better scalability in terms of energy consumption, where MariaDB is by far the worst energy-efficient \gls{dbms} with an increase of users, Postgres maintains similar energy consumption, MySQL and Redis improve their energy efficiency.

The other topic is energy consumption per performance scalability in both energy consumption per \gls{tpm} and energy consumption per \gls{nopm}, which has similar results, which Redis is who improves most, followed by MariaDB, Postgres, and MySQL.
\end{comment}