\chapter{Threats to Validity}
\label{cha:Threats}

This chapter is dedicated to threats that could endanger the quality of the presented results and conclusions. Here we present every threat to validity found in four categories as defined in \citeauthor{cook1979quasi}~(\citeyear{cook1979quasi}): 

\paragraph{Conclusion Validity}
%Resumo da categoria Conclusion Validity
In this category are the threats which may influence the capacity to draw correct conclusions~\cite{10.5555/2349018}.
    
%the reliability of measures.
Another threat of this category is the reliability of measures. This study has been made a comparison between the different consumption, and performance in every DBMS.  Also, we did comparisons between different subsystems like Package, DRAM, or secondary storage. Also, all DBMS execution and benchmark in a equal manner.  Every test was executed ten times and calculated the median, mean, standard deviation, min, and max values of these tests.

%Fishing (ter opçao de uma base de dados inves de outra)

Fishing for particular results can be an issue since the analyses are no longer independent~\cite{10.5555/2349018}. Here, it doesn't apply because this study isn't trying to promote any DBMS, and it isn't looking for any specific outcome.

%reliability of treatment implementation

Reliability of treatment implementation is another issue on conclusions valid. The software used was made by external developers. So, all the software made is independent of this study and was reused here to satisfy the study needs.

%random heterogeneity of subjects ( usamos apenas algumas base de dados)

The last threat is concerning random heterogeneity of subjects, while we only study four DBMS, this set includes some of the most popular ones and even different data models making. Here the only restriction was the HammerDB support.

%HAMMERDB E OS SEUS RESULTADOS e conclusions
Regarding the Hammerdb results conclusions, it was decided to use the TPM and NOPM to compare between each DBMS. This may not be the most suitable thing to do. While NOPM can be used to compare between different DBMS and is the optimal way to analyze DBMS, the same can't be said of TPM because it is not a metric recommended to compare different DBMS by the developers of the HammerDB ~\cite{hammerdb}. This happens because this value cannot be consistent between different databases. After all, every database reports different transaction rate metrics in their online tools.







\paragraph{Construct Validity}
% Resumo da categoria Construct Validity

Here are threats that involve the generalization of the results to the concept or theory behind the experiment~\cite{10.5555/2349018}.

%Inadequate preoperational explication
The first threat here is Inadequate pre-operational explication of constructs. This threat means that the constructs are not sufficiently defined before they are measured~\cite{10.5555/2349018}. Here, it is measure energy and HammerDB. So the measurements here were clear, making this issue minor or nonexistent here.


%Interaction of different treatments (nao tem ruido) e pausas entre execuçoes e testes de hammerdb
Another threat here is the interaction of different treatments is a possible issue related to if the subject is involved in more than one study that may interact~\cite{10.5555/2349018}. This does not happen here because we guarantee that only the minimum processes required to run each test is running. Additionally,  a two-minute idle time rest was essential to making the system to cool down to decrease overheating between each execution. This allows the system to treat garbage collecting and not affect the results with these actions. Also, every database has been analyzed in the same environment where each script was tested ten times, and every similar were equal in terms of workload configuration on HammerDB making all DBMS tested by the same rules.


%Mono-method bias
Mono-method bias is another issue. This issue is about the use of a single type of measure that involves a risk that the experiment could be misleading~\cite{10.5555/2349018}. Even though here we only use the same methods to measure energy. However, both are known to be very precise and reliable for measuring energy. So this doesn't affect the quality of the results.

%mono-method operation  
The last threat in this category is the mono-operation bios concern in that if the experiment includes only a single case may under-represent the construct and thus not give the full picture of the theory~\cite{10.5555/2349018}. This doesn't apply here because even though we only use one context, the context of a real environment is very reliable. After all, it represents an overall usage of a DBMS, making these results credible.





\paragraph{Internal Validity}

%resumo da categoria Internal validity

Threats to internal validity are influences that can affect the results of the study. Therefore, they endanger the conclusion of a possible association between treatment and outcome~\cite{10.5555/2349018}.

%Instrumentation and %Mediçao dos valores de escrever numa estrutura e %Asseguramos a sync dos processos

One of these threats is instrumentation. This threat is the effect of the artifacts used for the execution of the experiment~\cite{10.5555/2349018}. In this case, it was used scripts to collect the energy and execute HammerDB. Despite this, the scripts used on HammerDB are simple scripts to orchestrate the flow of SQL statements to the database to generate the required load. The other script used is to call software that executes energy measurement frameworks, where this software ensures that RAPL and Arduino are in sync, and saves in a struck in C and write the energy measurements inside it.  This can be seen as an overhead during measurements. But because this happens in every execution, this impact can be considered non-important. 



    

\paragraph{External Validity}

% Resumo da categoria external validaty

This category is concerned with the generalization of the results to industrial practice ~\cite{10.5555/2349018}.


%Interaction of selection and treatment.

Interaction of selection and treatment is the selection made not representative of the population wanted to generalize ~\cite{10.5555/2349018}. This doesn't apply in this study, because the databases selected are among the most popular on the market. Consequently, this applies only to an industrial environment in which such databases are used or intend to use.

%Interaction of setting and treatment

Another threat in this category is the interaction of setting and treatment, this is the effect of not having the experimental settings or materials representative of industrial practice~\cite{10.5555/2349018}.  In this study case, the benchmark tool HammerDB reproduces the industrial market because it is a well know open open-source tool used by different companies like Oracle, IBM, Intel, Dell/EMC HPE, Huawei, Lenovo, and hundreds more. Besides, the compilers, software versions, and computer used are recent and in line with nowadays industry. It is possible to adapt the Arduino to any industrial system, but it has to be calibrated. In the \citeauthor{portela2016} study, there is a calibration for Arduino used in this work. 





%Garantimos usar sempre o mesmo sistema e sempre a mesma versão das DBMS e generalização dos ddos e que o dados podem ser replicados
%As for results, One of the concerns was with the generalization of the results. The solutions obtained were the most effective at the time it was set up the analysis and because of that, it was decided to use the only one system, the same version of every DBMS and same version of HammerDB in all executions made. Measurements in different  systems and versions, might produce slightly different resulting values if replicated.


%FALAR SOBRE A SUBTRAÇÃO DOS RESULTADOS DO SISTEMA 
%To obtain only the dynamic energy consumption of the DBMS, first it was measured the operating system without any DBMS and HammerDB running and got a median of that energy consumption at every instance. With these values, we subtract the operating system values to each CSV of the \gls{rapl} and the Arduino. That is one of the reason why the DRAM results were insignificant and near-zero in this study and why sometimes it was chosen not to show the CPU graphs. 