%!TEX root = ../template.tex
%%%%%%%%%%%%%%%%%%%%%%%%%%%%%%%%%%%%%%%%%%%%%%%%%%%%%%%%%%%%%%%%%%%%
%% abstrac-pt.tex
%% NOVA thesis document file
%%
%% Abstract in Portuguese
%%%%%%%%%%%%%%%%%%%%%%%%%%%%%%%%%%%%%%%%%%%%%%%%%%%%%%%%%%%%%%%%%%%%

\begin{comment}
Independentemente da língua em que está escrita a dissertação, é necessário um resumo na língua do texto principal e um resumo noutra língua.  Assume-se que as duas línguas em questão serão sempre o Português e o Inglês.

O \emph{template} colocará automaticamente em primeiro lugar o resumo na língua do texto principal e depois o resumo na outra língua.  Por exemplo, se a dissertação está escrita em Português, primeiro aparecerá o resumo em Português, depois em Inglês, seguido do texto principal em Português. Se a dissertação está escrita em Inglês, primeiro aparecerá o resumo em Inglês, depois em Português, seguido do texto principal em Inglês.

O resumo não deve exceder uma página e deve responder às seguintes questões:
\begin{itemize}
% What's the problem?
	\item Qual é o problema?
% Why is it interesting?
	\item Porque é que ele é interessante?
% What's the solution?
	\item Qual é a solução?
% What follows from the solution?
	\item O que resulta (implicações) da solução?
\end{itemize}

E agora vamos fazer um teste com uma quebra de linha no hífen a ver se a \LaTeX\ duplica o hífen na linha seguinte…

zzzz zzz zzzz zzz zzzz zzz zzzz zzz zzzz zzz zzzz zzz zzzz zzz zzzz zzz zzzz comentar"-lhe zzz zzzz zzz zzzz 

Sim!  Funciona! :)
\end{comment}
Nos últimos anos, com o crescimento do consumo de energia pelos dispositivos computacionais, a eficiência energética é uma preocupação crucial na área de TI devido ao seu impacto económico e ambiental. Com a recente generalizada utilização de potentes dispositivos informáticos, nomeadamente smartphones, que dependem da “Cloud” para armazenar grandes quantidades de informação (como por exemplo, fotos e vídeos), está a exigir a construção e manutenção de grandes centros de dados. Esses centros de dados executam aplicações baseadas na Internet em grande escala, como serviços em nuvem. Como consequência, a energia consumida pela data centre está aumentar rapidamente, o que é uma preocupação crucial devido ao impacto económico e ambiental que estes trazem.

O aumento da dependência destes serviços em nuvem é uma das principais razões para o interesse em estudos e desenvolvimento de software e hardware com baixo consumo de energia. Hoje em dia, o uso mais popular dos data centre são os Sistemas de Gestão de Base de Dados (SGBD) que, normalmente, são responsáveis pelo acesso, gestão, manipulação e organização dos dados. Embora tenha havido alguns avanços e estudos em eficiência energética nesta área, ainda existe falta de conhecimento nesta área.

Esta dissertação pretende reduzir a falta de conhecimento do consumo de energia do software DBMS. Ao usar ferramentas de benchmarks que simulam ambientes reais, este estudo desempenha um papel fundamental no aprimoramento do conhecimento sobre a eficiência energética de diferentes tipos SGBD. Analisamos quatro sistemas, nomeadamente MySQL, Postgres, MariaDB e Redis. Além disso, usamos o framework de benchmark HammerDB para a simulação de SGBD em um ambiente real. Para ter um conhecimento aprefundado sobre o consumo de energia do SGBD, analisamos o consumo de energia em vários subsistemas do computador, nomeadamente como CPU, DRAM, GPU e Disco. Além disso, apresentamos uma análise mais aprofundada do consumo de energia relacionada com o desempenho em todos os níveis dos subsistemas. 
Esta tese apresenta resultados aonde pode ser verificado que existem diferenças significativas no consumo de energia das diferentes SGBD e em alguns cenarios, a Base de dados com melhor desempenho de performance de execução não é o que consome mais energia.






% Palavras-chave do resumo em Português
\begin{keywords}
Green Software, Green Computing, SGDBS, Eficiência Energética.
\end{keywords}
% to add an extra black line
