\begin{lstlisting}[ caption={ Exemple of reading RAPL energy in C},label={lst:raplcode},language=C,
    basicstyle=\tiny, %or \small or \footnotesize etc.
]
void rapl_after(FILE * fp , int core){ 
  int fd;
  long long result;
  fd=open_msr(core);

  result=read_msr(fd,MSR_PKG_ENERGY_STATUS);
  package_after=(double)result*energy_units;
  fprintf(fp,"%.6f , ",package_after-package_before);  // PACKAGE

  result=read_msr(fd,MSR_PP0_ENERGY_STATUS);
  pp0_after=(double)result*energy_units;
  
  fprintf(fp,"%.6f , ",pp0_after-pp0_before);    // CORE

  if ((cpu_model==CPU_SANDYBRIDGE) || (cpu_model==CPU_IVYBRIDGE) ||
	(cpu_model==CPU_HASWELL)) {
     result=read_msr(fd,MSR_PP1_ENERGY_STATUS);
     pp1_after=(double)result*energy_units;
     fprintf(fp,"%.6f , ",pp1_after-pp1_before);     // GPU
  }
  else fprintf(fp,"  , ");   
  
  if ((cpu_model==CPU_SANDYBRIDGE_EP) || (cpu_model==CPU_IVYBRIDGE_EP) ||
	(cpu_model==CPU_HASWELL)) {
     result=read_msr(fd,MSR_DRAM_ENERGY_STATUS);
     dram_after=(double)result*energy_units;
     fprintf(fp,"%.6f , ",dram_after-dram_before);     // DRAM
  }
  else fprintf(fp,"  , "); }
\end{lstlisting}
