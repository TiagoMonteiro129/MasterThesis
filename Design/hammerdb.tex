
We decide to use HammerDB as it is a benchmark tool that can replicate the behavior of most cloud-based applications as a service, verify comprehensive performance and multiple metrics in a simple real-world environment on a virtual environment with virtual users.~\cite{hammerdb}.



HammerDB is an Open source tool and accepted tool to benchmark DBMS because of this there is a lot of studies that use and recommend HammerDB to test and/or compare different DBMS ~\cite{scalzo2018database,elgrablyanalise,knoche2016combining,benchmarkchen,ali2019persistent,yu2015design,koccak2018software,koccak2018software}. \citeauthor{elgrablyanalise} study use HammerDB to compare open-source DBMS performance like Mysql, MariaDB and Postgres \cite{elgrablyanalise}. Another study that uses Hammerdb is \citeauthor{knoche2016combining}~(\citeyear{knoche2016combining}) work that use hammerdb to test the Impact of Database Lock Contention in the \gls{tpcc} Benchmark scenario \cite{knoche2016combining}. As a final example, we have \citeauthor{koccak2018software}~(\citeyear{koccak2018software}) that use HammerDB on MySQL to have a dataset to use on his software energy consumption prediction~\cite{koccak2018software}. 


HammerDB is also used by all leading database and technology companies. It has been downloaded hundreds of thousands of times to more than 180 countries in the world and even some well know companies such as Oracle, IBM, Intel, Dell/EMC HPE, Huawei, Lenovo, and hundreds more~\cite{hammerdb}.



HammerDB is a tool that can emulate a \gls{tpcc} scenario 
and through \gls{oltp} workloads it sets up a company's sales processing environment. It reduces the testing costs by simplifying the \gls{tpcc} rules, can be modified and run on a custom environment. The above factors result in a low-cost solution, rapid deployment, and personalized benchmark test \cite{benchmarkchen,elgrablyanalise,hammerdb}. 

 Since HammerDB implements a workload based on the \gls{tpcc} specification but does not implement a complete \gls{tpcc} benchmark specification, and because of that the transaction results from HammerDB can't be compared with the official \gls{tpcc} benchmarks published. HammerDB workloads generate 2 statistics. \gls{tpm} is the transactional measurement of the specific database typically defined as the number of user commits plus the number of user rollbacks. \gls{tpm} values are database-specific cannot be compared between different types of databases. The \gls{nopm} value, on the other hand, is a performance metric independent of any particular database implementation and is the recommended primary metric to use~\cite{hammerdb}.

HammerDB currently supports Oracle, SQL Server, Db2, TimesTen, MySQL, MariaDB, PostgreSQL, Greenplum, Postgres Plus Advanced Server, Redis, and more, and can run on a variety of operating systems, being an elastic and heterogeneous tool~\cite{benchmarkchen}.

Another reason we choose him was that HammerDB is used for automated software testing \cite{hammerdb} so we can configure how many times the HammerDB will run and how many virtual users that he will use and for how long it will run, providing at the end of every execution the \gls{tpm} and the \gls{nopm}. With these numbers and the energy spent, we can have ample information to answer the RQ2, and with an adjustable number of users, it can assist in the RQ3.

